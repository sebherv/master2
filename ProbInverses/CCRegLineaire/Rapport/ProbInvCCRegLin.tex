\documentclass[12pt,a4paper]{extreport}
\usepackage[utf8]{inputenc}
\usepackage[french]{babel}
\usepackage[T1]{fontenc}
\usepackage{amsmath}
\usepackage{amsfonts}
\usepackage{amssymb}
\author{Sébastien Hervieu}
\title{Contrôle Continu Problèmes Inverses: Régression Linéaire}
\renewcommand{\thesection}{\alph{section}}
\begin{document}
\maketitle

\section{Equation Linéaire}

Nous disposons de données $d(t,y)$ contenant les hauteurs $y$ du projectile à des instants $t$.
Par ailleurs, le modèle que nous voulons trouver est de la forme:
\[y(t)=m_1+m_2t-\frac{1}{2}m_3t^2\]

Nous posons donc l'équation suivante:
\begin{gather}
d = Gm
\end{gather}

et donc 

\begin{gather}
	\begin{bmatrix} y_1 \\ y_2 \\ \vdots \\ y_{25} \end{bmatrix}
	=
	\begin{bmatrix}
	1 & t_1 & -\frac{1}{2}t_1^2 \\
	1 & t_2 & -\frac{1}{2}t_2^2 \\ 
	\multicolumn{3}{c}{$\vdots$} \\
	1 & t_{25} & -\frac{1}{2}t_{25}^2
	\end{bmatrix}
	\begin{bmatrix} m_1 \\ m_2 \\ m_3 \end{bmatrix}
\end{gather}
où les $y_i$ et les $t_i$, $(i= 1, \dots, 25)$  sont les données et les $m_n$, $(n =1, 2, 3)$ sont les inconnues.


\section{Solution des moindres carrés, matrices de covariance et de corrélation}

La solution des moindres carrés est donnée par :
\begin{equation}
	\begin{gathered}
		m = (G^TG)^{-1}G^Td
	\end{gathered}\label{eq:1}
\end{equation}

En résolvant numériquement à l'aide de scilab, nous obtenons:
\begin{equation}
\begin{array}{c}
m_1 = 21396.743 \\[\jot] m_2 = 16.811659 \\[\jot] m_3 = 3.7141271
\end{array}
\end{equation}

\subsection*{Matrice de covariance des données}
En utilisant la formule donnée en \eqref{eq:1} pour le calcul des paramètres, nous faisons l'hypothèse implicite que l'erreur sur les données $\sigma$ est telle que 
\begin{gather}
\sigma^2 = 1
\end{gather}
Dans cette hypothèse la matrice de covariance des données $C_d$ est égale à la matrice identité $I$, soit:

\begin{gather}
	C_d =
	\begin{bmatrix}
	\sigma_1^2 & 0 & \dots & 0 \\
	0 & \sigma_2^2 & \dots & 0 \\
	\multicolumn{3}{c}{$\vdots$} \\
	0 & 0 & \dots & \sigma_{25}^2
	\end{bmatrix}
	=
	\begin{bmatrix}
	1 & 0 & \dots & 0 \\
	0 & 1 & \dots & 0 \\
	\multicolumn{3}{c}{$\vdots$} \\
	0 & 0 & \dots & 1 
	\end{bmatrix}
	= I
\end{gather}

\subsection*{Matrices de covariance du modèle}
La matrice de covariance du modèle $C_m$ est donnée de manière générale par:
\begin{gather}
	C_m = (G^TC_d^{-1}G)^{-1}
\end{gather}

Dans notre hypothèse où $C_d = I$, $C_m$ devient:
\begin{gather}
	C_m = (G^TG)^{-1}
\end{gather}

$C_m$ est calculée à l'aide de scilab; nous obtenons le résultat suivant:

\begin{gather}
	C_m =
	\begin{bmatrix}
		0.3654177 & -0.0120938 & -0.0001784 \\
  		-0.0120938 &  0.0005567 &  0.0000097 \\
  		-0.0001784 &  0.0000097  & 0.0000002
	\end{bmatrix}
\end{gather}

La matrice de corrélation des paramètres est donnée par la formule suivante:
\begin{gather}
	\rho_{ij} = \frac{C_m(m_i,m_j)}{\sigma_i\sigma_j}
\end{gather}
ce qui nous donne dans notre cas:
\begin{gather}
	CorrM =
	\begin{bmatrix}
	1 &-0.8479488 & -0.6838399 \\
  -0.8479488  & 1.        &  0.9535659 \\
  -0.6838399 &  0.9535659  & 1.  
	\end{bmatrix}
\end{gather}

\subsection*{Interpretation}
Nous constatons que $Cm$ n'est pas diagonale, qu'aucune valeur de la matrice de $CorrM$ n'est nulle.

\section{Estimation des paramètres avec $\chi_{\nu=1,p=0.95}^2$}
En effectuant le calcul suivant:
\begin{gather}
	\Delta m = \sqrt{\chi_{\nu=1,p=0.95}^2 \times diag(C_m)}
\end{gather}

avec $\chi_{\nu=1,p=0.95}^2 = 3.96$


nous obtenons les estimations des paramètres suivantes, avec leurs intervalles de confiance à 95\% projetés en 1D:

\begin{equation}
	\begin{array}{c}
		m_1 = 21396.743 \pm 1.2029356\;m\\
		[\jot] m_2 = 16.811659 \pm 0.046951\;m/s\\
		[\jot] m_3 = 3.7141271 \pm 0.0008587\;m/s^2
	\end{array}
\end{equation}

\section{Estimation des paramètres avec $\chi_{\nu=2,p=0.95}^2$}
En effectuant un calcul similaire au précédent:

\begin{gather}
	\Delta m = \sqrt{\chi_{\nu=2,p=0.95}^2 \times diag(C_m)}
\end{gather}

avec $\chi_{\nu=2,p=0.95}^2 = 5.99$


nous obtenons les estimations des paramètres suivantes, avec leurs intervalles de confiance à 95\% 3D projetés en 2D:

\begin{equation}
	\begin{array}{c}
		m_1 = 21396.743 \pm 1.4794769\;m\\
		[\jot] m_2 = 16.811659 \pm 0.0577445\;m/s\\
		[\jot] m_3 = 3.7141271 \pm 0.0010561\;m/s^2
	\end{array}
\end{equation}

\end{document}